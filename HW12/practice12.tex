\documentclass{article}

\usepackage{courier}
\usepackage{graphicx} % Required for the inclusion of images
\usepackage{listings}
\usepackage{color}
\usepackage{amsmath}
\usepackage{subcaption}
\usepackage{enumitem}
\usepackage{float}
\usepackage[toc,page]{appendix}
\usepackage{dcolumn}
\usepackage{pdflscape}
\usepackage{hyperref}
\usepackage{framed}
\usepackage[english]{babel}
\usepackage{soul}
\usepackage{caption}
\usepackage{amsfonts}
\usepackage{MnSymbol,wasysym}
\usepackage{algorithmic}
\usepackage{mathtools}

\DeclarePairedDelimiter\abs{\lvert}{\rvert}%
\DeclarePairedDelimiter\norm{\lVert}{\rVert}%
\captionsetup[figure]{labelformat=empty}%

\setlength\parindent{0pt} % Removes all indentation from paragraphs

\renewcommand{\labelenumi}{\alph{enumi}.} % Make numbering in the enumerate environment by letter rather than number (e.g. section 6)

\definecolor{dkgreen}{rgb}{0,0.6,0}
\definecolor{gray}{rgb}{0.5,0.5,0.5}
\definecolor{mauve}{rgb}{0.58,0,0.82}

\lstset{frame=tb,
  language=R,
  aboveskip=3mm,
  belowskip=3mm,
  showstringspaces=false,
  columns=flexible,
  basicstyle={\small\ttfamily},
  numbers=none,
  numberstyle=\tiny\color{gray},
  keywordstyle=\color{blue},
  commentstyle=\color{dkgreen},
  stringstyle=\color{mauve},
  breaklines=true,
  breakatwhitespace=true
  tabsize=3
}

\newcommand{\footlabel}[2]{%
    \addtocounter{footnote}{1}%
    \footnotetext[\thefootnote]{%
        \addtocounter{footnote}{-1}%
        \refstepcounter{footnote}\label{#1}%
        #2%
    }%
    $^{\ref{#1}}$%
}

\newcommand{\footref}[1]{%
    $^{\ref{#1}}$%
}

%\usepackage[colorlinks]{hyperref}
\hypersetup{linkcolor=DarkRed}
\hypersetup{urlcolor=DarkBlue}
\usepackage{cleveref}

\title{Data Mining\\Homework Assignment \#12} % Title

\author{Dmytro Fishman, Anna Leontjeva and Jaak Vilo} % Author name

\begin{document}

\maketitle % Insert the title, author and date 

\section*{Task 1}
State why for the integration of multiple heterogeneous information sources many
companies in industry prefer the update-driven approach (which constructs and uses
data warehouses), rather than the query-driven approach (which applies wrappers and
integrators). Describe situations where the query-driven approach is preferable over
the update-driven approach.

\section*{Task 2}
A data warehouse can be modeled by either a star schema or a snowflake schema.
Briefl�y describe the similarities and the differences of the two models, and then
analyze their advantages and disadvantages with regard to one another. Give your
opinion of which might be more useful on practice and state the reasons behind
your answer.

\section*{Task 3}
Explain the terms slice and dice, drill-down and roll-up. Provide examples and/or illustrations
 
\section*{Task 4}
Suppose that a data warehouse consists of the four dimensions: date, visitor, location, and movie, and the two measures, count and charge, where charge is the fare that
a visitor pays when watching a movie on a given date. Visitors may be students,
adults, or seniors, with each category having its own charge rate.
\begin{enumerate}
\item Draw a star schema diagram for the data warehouse.
\item Starting with the base cuboid [date, visitor, location, movie], what specifi�c OLAP
operations should one perform in order to list the total charge paid by student
visitors at Cinamon Cinema in 2004?
\end{enumerate}

\section*{Task 5}
Rewrite the following query by replacing GROUP BY CUBE(...) clause into an equivalent query using UNION and simple GROUP BY:
\begin{verbatim}
SELECT product, year, city, sum(price*vol)
FROM Orders
GROUP BY CUBE(product, year, city);
\end{verbatim}

\section*{Task 6}
\end{document}
